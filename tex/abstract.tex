%# -*- coding: utf-8-unix -*-
%%==================================================
%% abstract.tex for SJTU Master Thesis
%%==================================================

\begin{abstract}
    随着互联网行业的发展,如今的互联网正处于一个信息爆炸的时代。IaaS(Infrastructure as a Service)云常常被用来部署大数据分析和其他各种类型的应用。
    虚拟机被认为构建IaaS云的基础,而虚拟机常常是基于虚拟机磁盘镜像(VMDI)构建的。对象存储可以让云提供商轻松地扩大存储规模,并且它通过
    备份数据副本的方式保证了存储的高可用性。存储介质方面,SDD在性能、能耗等方面相较于HDD有较大的优势,但是由于其高昂的价格与存储空间,让SSD完全提成HDD成为企业级
    数据存储还是太现实。

    WHOBBS\cite{lingxuan2015whobbs}是本实验室之前关于混合存储的研究,它基于对象存储为虚拟机提供混合存储的块服务。它在Ceph\cite{weil2006ceph}的基础上,
    构建了一套完整的混合存储系统,但是WHOBBS由于其单监控器架构,导致监控器成为系统的性能负担。本文设计和实现DOBBS系统,它在WHOBBS的基础上进行了扩展,DOBBS
    为多监控器架构,并将整个集群分割成多个子集群。为了充分利用SSD和HDD混合存储的高效性,我们提出了局部热均衡的概念。在分割成子集群之后可能遇到子集群间热度分布不均衡的
    情况,我们提出了一个全新的叫做全局热均衡的概念。全局热均衡主要包括热度不均衡的检测和热扩散散。在热度不均衡检测部分,我们提出了衡量子集群热度的方法以及热度检测的算法。
    热扩散是全局热均衡的重点,为了解决大规模数据迁移,我们提出了全新的源自于物理学的概念——热扩散。我们只对元数据进行迁移,并且修改子集群的逻辑结构,而后续的数据迁移则是
    交给局部热均衡来完成的。

    在系统设计的基础上,我们对系统加以实现,并介绍了DOBBS的各个模块的实现细节和主要工作流。在文章的最后,为了验证DOBBS的高效性,我们进行了大量的实验。
    大量实验证明,DOBBS解决了WHOBBS的单监控器的性能问题,同样也验证了全局热均衡的有效性。


\keywords{\large 分布式存储 \quad 混合存储 \quad 负载均衡}
\end{abstract}

\begin{englishabstract}
    With the development of the Internet industry, the Internet today is in an era of information explosion. IaaS(Infrastructure as a Service)
    are often used to deploy big data analytics and other types of applications. Object storage allows cloud providers to easily scale up their storage, 
    and it ensures high availability of storage by backing up data copies. In the aspect of storage media, SSD in performance, energy consumption and other 
    aspects compared to HDD has greater advantage, because of its high price and limited capacity, it is still a long way to go before completely using SSDs 
    for enterprise data storage.

    WHOBBS\cite{lingxuan2015whobbs} is our previous research on hybrid storage. WHOBBS, while effective, only supports single monitor architecture as it 
    concentrates all the monitoring tasks on one single node. Thus, single monitor would be a serious problem for WHOBBS. In this thesis, we design and implement
    DOBBS, which extends WHOBBS with multi-monitor architecture and splits the cluster into subclusters. We put forward the concept of Local Heat Balancing for 
    managing the system efficiently.Local heat balancing, actually, indicates fully exploiting the unique performance potential of SSD, and ensures high accessed data 
    reside on SSDs. We propose a concept named Global Heat Balancing which ensures the whole cluster is in the state of heat balance. Global Heat Balancing includes two parts,
    heat imbalance detection and heat diffusion. 

    Based on system design, we implement the system and introduce the implementation details and the main workflows of each module of DOBBS. In order to verify the efficiency of 
    DOBBS, we conducted a large number of experiments. A large number of experiments show the efficiency of DOBBS and the effectiveness of Global Heat Balancing.

\englishkeywords{\large Distribued Storage, Hybrid Storage, Load Balance}
\end{englishabstract}

