%# -*- coding: utf-8-unix -*-
%%==================================================
%% chapter01.tex for SJTU Master Thesis
%%==================================================

%\bibliographystyle{sjtu2}%[此处用于每章都生产参考文献]
\chapter{系统设计}
\label{chap:systemdesign}
DOBBS是为了适应大规模IaaS云的底层存储,利用对象存储对非结构化数据VMDI进行大量优化,并且在WHOBBS的基础上解决了其性能缺陷。本章我将介绍DOBBS的
系统架构设计以及各模块设计。

\section{DOBBS架构介绍}
所谓混合存储系统,就是将SSD与HDD共同组合而成的存储系统。DOBBS的系统设计是基于Ceph实现的,而Ceph的文件实际是存储在对象存储设备中的(Objects 
Storage Device)。为了提升DOBBS系统的高扩展性,我们借用了Ceph存储池(Storage Pool)的概念。存储池实际是部分对象存储设备的逻辑集合,一个存储池
可以包含多个对象存储设备,而对象存储设备是挂载在物理存储介质上的。因此对于DOBBS,我们有两种类型的对象存储设备,分别是SSD对象存储设备和HDD对象存储设备。
之后,我们将部分SSD对象存储设备和部分HDD对象存储设备共同组织成一个混合存储池(Hybrid Storage Pool)。混合存储池中的对象存储设备是可以由用户配置的,
通过这样的组织,DOBBS的底层存储集群可以简单高效地扩展。一般性的,每个混合存储池都至少包括一个SSD对象存储设备和至少一个HDD对象存储设备。


\begin{figure}[!htp]
    \centering
    \includegraphics[width=16cm]{example/Arch.pdf}
    \bicaption[fig:arch]{DOBBS系统架构图}{DOBBS系统架构图}{Fig}{Architecture of DOBBS}
   \end{figure}


如图\ref{fig:arch}所示为DOBBS的系统架构图。DOBBS的架构大致可以分为两层,一层是存储集群也就是混合存储池,一层是监控器(Monitor)。DOBBS包含大量
的混合存储池以及监控器。存储集群正如上文所说的,它是有大量的混合存储池组成的。为了系统的高可靠性和高性能,我们将一个监控器和一个混合存储池绑定在一起,组成一个子
集群(subcluster)。之所以要划分成子集群是因为多个Monitor如果同时管理所有的混合存储池势必会造成管理的混乱,并大大降低系统的可扩展性。引入子集群则是可以充分
提升系统的可扩展性,并使整个系统易于管理。监控器在DOBBS则是起到至关重要的作用,它的作用包括监控虚拟机的数据流信息并生成合适的迁移策略,还有检测其所在子集群的热度并向中心节点(Center)
汇报,最终执行全局热均衡迁移指令。在DOBBS中,子集群仅仅只是逻辑上的概念,虚拟机VMDI则被均匀的分布在各个子集群的混合存储池中。监控器所监控的数据流信息也仅仅是其所在
子集群对应数据的数据流信息。客户端(Client)则是虚拟机管理器(VMM)所运行的机器,它们是DOBBS所提供服务的对象,客户端在接入DOBBS之后,与底层存储系统交互并进行数
据读写。中心节点则是DOBBS的监控器集群的大脑,它的主要作用是收集每个子集群的热度信息,并保证整个系统的热均衡。关于整个系统的热均衡,在本章后续章节将会叙述。

为了系统的高内聚低耦合,我们将系统分为多个模块,模块间互相依赖并支持高效扩展。在客户端上,如果让虚拟机管理器直接访问Ceph的块存储,那么它只需要调用Ceph提供的
OSD Component接口即可,但是为了获得虚拟机的数据流信息我们VM与Ceph存储设备的交互中间增加了一层DOBBS监控组件。因为Ceph是对象存储系统,所以虚拟机访问的粒度都是
一个个对象。如图\ref{fig:arch}所示,IO Controller负责截取VM的访问请求,然后通过查询Object Table来知道对象是在哪个OSD上,这是因为在数据迁移的过程中对象被频繁迁移,
而Ceph访问对象是通过对象ID和OSD共同组合访问,因此需要一个数据结构在存储对象ID和OSD ID的映射。Gather模块类似于一个计数器,它会记录每一个单位时间间隔内的
对象的访问次数、访问类型等信息,在记录之后再定时发送给Monitor。Monitor Table则是用来维护OSD ID和子集群ID映射的,因为在DOBBS在保证整个系统热均衡的过程中需要对OSD上的数据进行跨子集群迁移,所以
可能会导致某个OSD的位置发生了改变。最终,Ceph根据VM请求的对象ID和OSD ID来和底层存储进行交互。

Monitor主要有六个模块。Analyzer模块是Monitor的大脑,它负责接收客户端传送过来的虚拟机数据流信息,然后根据内置算法来生成合适的对象迁移命令,命令产生后将迁移命令传递给
Migrator模块执行。Migrator得到迁移指令之后,进行后续的上锁等等操作,然后再将最终的指令缓存在Migrator Queue中,这个队列的长度是DOBBS所支持的最大迁移数量,这个数量
当然也是可配置的。在Monitor上还有三个独立的模块,Reporter、HeatCollector和Blancer,他们都是用来负责系统热均衡的。

DOBBS提出了两个重要的概念,局部热均衡和全局热均衡。对于局部热均衡,我们数据对象分为两类,热对象和冷对象。所谓热对象是指的那些访问频率高,短时间内会被多次访问的对象。相反,
冷对象则是指那些访问频率并不高的对象。为了充分利用SSD的性能优势,我们将热对象都放置于SSD上而将冷对象都放置于HDD上,然后动态地监测数据对象的冷热变化来在SSD和HDD之间迁移对象。
局部热均衡只是针对每个子集群内部的局部热均衡,因此每个Monitor只能保证所在子集群的局部热均衡。如果将存储集群和Monitor分割成多个子集群,那么极有可能在
某个时刻某个虚拟机产生了非常大量的VMDI请求,这样某个子集群会遭受大量的读写请求,数据也会频繁地在SSD和HDD间迁移,这个子集群的效率将会受到极大影响。提出全局热均衡的概念就是为了
消除上述情况带来的性能波动。全局热均衡针对的粒度是OSD级别的,它会动态监测每个子集群的热度,然后做出迁移指令,有别于局部热均衡的迁移,全局热均衡的迁移是将OSD的数据进行迁移而不是
在OSD进行数据对象的迁移。

\section{局部热均衡}


