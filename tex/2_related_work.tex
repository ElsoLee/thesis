%# -*- coding: utf-8-unix -*-
%%==================================================
%% chapter01.tex for SJTU Master Thesis
%%==================================================

%\bibliographystyle{sjtu2}%[此处用于每章都生产参考文献]
\chapter{国内外研究现状}
\label{chap:relatedwork}
随着云计算的高速发展,每天在互联网中都会产生海量的数据。分布式存储作为云计算的基础,近些年有大量的关于分布式存储的研究。
以下将通过:分布式存储及对象存储、混合存储的应用以及分布式存储负载均衡等几个方面来进行国内外研究现状的综述。

\section{分布式存储及对象存储}
分布式存储作为云计算的基础,谷歌在2003年提出的GFS\cite{ghemawat2003google}是在分布式系统领域具有划时代意义的论文,它的提出了如何采用
廉价的商用计算机集群构建分布式文件系统,论文中提出的控制与业务相分离、将容错的任务交由文件系统来完成、多副本机制等等,这些概念大多都变成了
之后分布式存储系统设计研究的金科玉律。在随后的Apache的HDFS\cite{shvachko2010hadoop},在一些设计上与GFS高度一致,它提供了一个高度容错的分
布式文件系统,能够提供高吞吐量的数据访问。

对象存储的概念在上个世界90年代就应该在研究社区中出现,但是在当时由于云计算还没有广泛流行,所以对象存储也没有被广泛运用。对象存储,是将对象作为单位管理数据,与之对应的
文件存储则是以文件系统结构为基础,而块存储则是以块为单位管理数据。\citen{factor2005object}介绍了对象存储发展的几个阶段,首先是由TIO OSD协议作为第一阶段的标准,
第二阶段则是iSCI。Ceph\cite{weil2006ceph}作为当今非常流行并且成熟的分布式对象存储,已经集成到了Linux内核中。Ceph是一个基于对象存储的PB级分布式文件系统,
拥有极佳的可靠性和可扩展性。根据配置的不同,Ceph可以分别提供上述的块存储、文件存储和对象存储三种服务,但是所有的服务均基于一个叫Rados的对象存储系统。

\section{混合存储的应用}
SSD的读写性能、能耗等各个方面较于传统的HDD有较大的优势,因此近些年来SSD被越来越多的用于计算机存储领域。\citen{kasavajhala2011solid}中揭示了SSD和HDD性能上的差距,
并且建议用户通过价格和性能的比率来选择合适的存储设备,对于不同的数据流SSD和HDD在性能上的表现也有一定的差别。对于顺序数据流,SSD在性能上的提升相较于其高昂的
价格就并没有太多的优势,而对于随机数据流SSD对于读写性能的提升十分显著。因此考虑到SSD高昂的价格,以及其在特定数据流表现的性能差异,将SSD与HDD作为存
储设备混合使用将会是一个研究的重点。

对于关于SSD/HDD混合存储的研究主要分为两类,一类是考虑到SSD相较于HDD具有节能、高效的特性,将其用作存储缓存;另一类是考虑到SSD较高的读写性能、非易失性以及拥有大容量,
将SSD作为主要二级存储。以下将从这两类介绍SSD在混合存储中的应用。

在\citen{wan2014ssd}中,作者提出了一个混合式的对象放置和移动架构,以及适应性学习算法来识别数据对象的访问热度。该论文中,混合式架构通过将数据对象在慢速的HDD和快速的SSD间相互移动来适应
数据对象访问热度的动态变化,同时数据对象的放置及移动规则也可以满足用户对它们I/O操作的需求。基于以上的目标,论文提出了一个基于马尔科夫链的识别模型来预测数据对象在外来一段
时间的访问频率,这个识别模型是以对象的访问模式为训练集训练马尔科夫链模型,一旦的得到预测结果就开始在HDD和SSD间迁移数据。为了适应用户自定义的放置规则,论文提出了一个数据放
置引擎,以用户的放置需求作为输入产生合适的解决方法。\citen{wan2014ssd}的核心思想就是,作者将SSD和HDD的混合存储用到了高性能计算领域(HPC),因为是它基于预测的思想,所以可以在整个系统数
据发生变化之前就将数据进行迁移,这样大大提高了整个系统的IO性能。

\citen{li2015trade}将目光放在了对SSD/HDD混合存储在性能、耗能和寿命问题做出了研究。\citen{li2015trade}设计并实现了一个多用的混合存储驱动,它实现的混合存储系统包含两个模式——分层存储模式和缓存模式,
这两个模式分别将SSD用作存放热点数据的主存和用于存储热数据的缓存。该系统从混合虚拟设备接受数据访问请求,并将结果重定向给下层物理设备。而\citen{li2015trade}在区分冷热数据的方式与\citen{wan2014ssd}相
似,也就是将热的、访问频率高的数据放在SSD中,反之冷的、访问频率低的数据放在HDD中,当数据的冷热情况发生变化,再动态地将数据进行迁移。但是它把研究的重心放在了对性能、耗能以及寿命的权衡。它得出的结论是,
为了得到高性能,势必会导致能耗的增加;对于论文中,数据分块越大性能提升越大,同时会造成SSD寿命的快速衰减。\citen{wan2014ssd}和\citen{li2015trade}都分别研究了以SSD/HDD为主体的混合存储,并且研究的重
点都在于利用SSD在读写方面优异性能,将热点数据存储在SSD中,冷的数据存储在HDD中,再通过这一基本框架做进一步的研究。

而\citen{wan2017optimizing}的研究重点面向了更加特定的环境,即面向高性能计算环境下基于SSD突发缓存的研究。\citen{wan2017optimizing}的研究不仅限于SSD/HDD这两种局限的存储介质类型,而是更加抽象的存储分层,它以SSD作为突发缓存,
而底层存储是并行文件系统(PFS),突发缓存作为客户端与并行文件系统之间的缓存,常常要存储检查点(checkpoint),而检查点是需要频繁读写的,这样会造成SSD的寿命大幅缩短,并且由于高性能计算环境数据流往往是高速、大流量的,所以对作为突发缓存的硬件寿命
面临着极大的挑战。基于这个问题,\citen{wan2017optimizing}设计了一个适应性算法基于高性能计算环境下系统数据流的变化,动态的切换检查点的放置策略,这样既保证了系统的高效运行也大大延长了突发缓存的硬件寿命。

Apple公司在2012年发布了自己的混合存储设备Fusion Drive,并将其应用到Apple的各个产品中。Fusion Drive融合了HDD和SDD,系统会自动管理访问最频繁的应用程序、文件、照片或者其他数据来存储在SSD里,而将很少访问或使用的文件留在机械硬盘上。
\citen{chen2011hystor}为Apple开发Fusion Drive提供了思路。在\citen{chen2011hystor}中,作者设计并实现了一个高性能混合存储系统Hystor。作者在论文中提出了混合存储必须面对的三个问题,分别是1)高效地识别性能关键的数据快并充分利用
SSD的存储潜力;2) 为了精确地刻画数据访问模式必须高效地保存数据访问历史;3) 系统实现的过程中避免对现存操作系统内核的改变。\citen{chen2011hystor}实现的Hystor充分解决了上述的三个问题,它通过监控I/O自动地学习数据流访问模式并识别出
性能关键的数据块,只有带来最大性能收益的数据块才可以从HDD重映射到SSD中;其次,Hystor设计了一个高效的机制即数据块表(Block Table)去存储数据块访问的历史信息,这样可以为长期的优化提供资源;它是一个Linux的内核模块,只对内核做了很少的修改。
因此\citen{chen2011hystor}作为一个具有里程碑意义的论文,颠覆了以前仅仅利用SSD优秀读写性能而将SSD作为缓存的传统模式,而提出了将SSD和HDD混合做个主要存的新模式。

对于\citen{krish2014hats},它将混合存储用到了HDFS(Hadoop Distributed File System)中,由于原有的HDFS并不能用来处理多层存储,。HDFS认为所有底层存储组件都具有相同的I/O特性,这样会造成对资源的浪费和不高效实用。因此,在\citen{krish2014hats}中,
作者设计并实现了一个针对Hadoop的多层存储系统hatS,它在逻辑上把相同存储类型的节点分到同一个层,这样单独地管理各个存储层,hatS就能够捕捉分层信息并充分利用这些信息从而达到高的I/O性能 。与HDFS不同的是hatS使用分层感知的方法在不同之间复制数据,这样使得
高性能存储设备可以高效地处理更加处理大量I/O请求。hatS的实际做法是利用Hadoop的多数据备份的规则,并结合存储分层,让尽量多的I/O请求作用到性能高的节点上,从而提升整个系统的吞吐能力。

\citen{ma2014mobbs}\citen{lingxuan2015whobbs}两篇论文,将混合存储和当今非常流行的对象存储系统Ceph相结合,\citen{ma2014mobbs}设计并实现了基于Ceph的混合存储系统WHOBBS。由于原始的Ceph系统并没有对底层存储介质的类型给予优化,它是根据CRUS
H算法\cite{weil2006crush}静态地设置osd权值,根据权值放置数据,这个过程是静态的,并不能使用数据访问热度的动态变化。因此WHOBBS动态监测每一个数据对象,并且保存数据对象的历史访问记录,根据用户自定义算法来指定迁移策略,即SSD与HDD之间的迁移策略。在设
计迁移策略之前,\citen{lingxuan2015whobbs}做了大量实验验证SSD与HDD在Ceph不同数据流下的性能,并根据实验结果指定合理的数据对象迁移策略。

综上所述,现今关于混合存储的研究大多都着眼于利用SSD优秀的存储性能以及低能耗等特点,但是由于SSD高昂的价格、有限的寿命以及在部分数据流方面表现的差异,所以现在仍用很多数据存储公司权衡这方面的利弊,选择了混合存储架构。如现在的Apple FusionDrive、
Microsoft公司的Ready Drive、西部数据公司的SSHD等等。而这些产品大多都是面向单机存储的,它们并没有充分应用到分布式存储的环境中,正如\citen{krish2014hats}所说,现在的大部分的分布式计算框架对待底层存储都没有根据存储的类型来进行优化。

\section{分布式存储负载均衡}
在大规模云计算环境中,负载均衡技术通过保证每一个计算资源可以获得公平的计算请求,达到了用户满意度,同时提高了资源利用率。合适的负载均衡目技术为的就是减小资源竞争、保证可扩展性、避免性能瓶颈等等。

Paiva J.等人在2015年提出了AutoPlacer\cite{paiva2015uto},它面向现今最流行的键值存储提供数据放置策略,AutoPlacer的自调节数据放置策略可以较好地达到负载均衡目标。AutoPlacer使用了一个轻量级的分布式优化算法
,这个算法不断迭代,在每一次迭代中它以一种去中心化的方法优化前K个热点数据的放置。热点数据实际上产生最多远程操作的数据对象。为了识别出热点数据,并且不会有对整个系统的性能带来太大的影响,AutoPlacer
选用了一种全新流分析算法来追踪数据流(data stream)中前K个最常被访问数据对象。AutoPlacer不仅支持细粒度的数据放置,还使用了全新的概率数据结构PAA来保证单跳路由的延迟最小。


\citen{zhang2015skewly}提出了一个全新的想法,它使用倾斜的复制策略来构建节能存储集群。它提出的数据复制策略实际是基于数据访问行为的。这个策略仅仅复制访问热点数据,而这些热点数据根据80/20原则(80\%
的数据访问往往只访问20\%存储空间的数据)恰恰只是整个数据齐群20\%的数据量。因此,可以将集群分为热节点集和冷节点集。热节点存储较少数量的数据副本,并且常常处于活跃状态来保证整个系统的QoS
。冷节点则与之相反,它存储数量较大且访问频率较小的冷数据,并且这些存储节点被通过一定的方式(CPU降频、磁盘降低转速)来进入节点模式。作者在论文中通过模式实验中验证了自己的观点,其所提出的系
统不仅可以保证低能耗也可以保证整个系统的性能不受到影响。

Soni G.等人另辟蹊径提出了全新的云数据中心的负载均衡方法\cite{soni2014novel}。这个负载均衡方法面向云数据中心的虚拟机数据流。每一个来自用户的请求都从数据中心控制器发起,数据中心控制器遍历中心负载均衡器,通过遍历结果来分配请求。
中心负载均衡器会维护一个包含id、虚拟机状态以及虚拟机优先级的数据表。中心负载均衡器分析数据表并且找到最高优先级的虚拟机,然后检查其状态,如果它的处于可用的状态则返回虚拟机id给数据中心控制器。最终,数据中心控制器
将会把某个数据请求分配给特定的虚拟机。作者通过实验验证了中心负载均衡方法的可行性。

Deng Y.等人提出的负载均衡技术\cite{deng2014dynamic}进行了创新,他们利用了物理学的热扩散原理来处理分布式虚拟环境(DVE)的负载均衡问题。在论文中,作者使用了一种混合式的负载均衡策略,也就是将全局负载均衡和局部负载均衡相融合。全局负载均衡指的是,
存在一个中心节点来管理所有服务器的负载情况,它从各个服务器收集负载数据,并制定负载策略,这样做的优点就是可以准确地实现负载均衡,但是计算量较大,需要额外的计算资源。与全局负载均衡相反,局部负载均衡则不存在中心节点,每个服务器通
过其邻接的服务器负载数据来制定自己的负载均衡策略,这样做的优点在于计算量较小,但是负载的结果不并准确。在\citen{deng2014dynamic}中,作者使用了热扩散原理,即高温区域的热量会向低温区域扩散,来处理负载均衡问题,并且将热扩散原理与全局负载均衡和局部负
载均衡相结合,提出了局部扩散和全局扩散。对于热扩散在负载均衡中的应用,高度概括为,过热服务器将自己的部分数据流交给邻接较冷的服务器去处理。作者通过大量的实验和数学证明验证了基于热扩散的负载均衡技术的高效性。

综上所述,现在已经有许多在分布式系统负载均衡方面的研究,但是可以发现一点,就是这些\citen{paiva2015uto}\citen{zhang2015skewly}\citen{deng2014dynamic}关于负载均衡的研究都是相对静态的,并且大多是关于数据放置如何达到负载均衡,
并没有考虑计算节点过热时如何动态均衡计算任务。

\section{本章小结}
本章通过分布式存储和对象存储、混合存储和分布式存储负载均衡,这三个方面介绍了与本课题相关的国内外研究现状。分布式存储和对象存储介绍了,现在主流的分布式存储框架以及Ceph等对象存储系统,并叙述了这些系统
在特点及应用场景。混合存储部分,从SSD和HDD的性能差距着手阐述了混合存储的必要性和意义,之后分别介绍了国外关于混合存储的研究。混合存储的主要研究还是主要面向于将SSD运用为主存,或是将SSD运用为缓存。但是现在
关于混合存储的研究大部分还是面向单机存储的,分布式存储的应用相对较少。分布式负载均衡部分,介绍了多种分布式存储负载均衡的解决方案,但是可以清楚地看到,大部分解决方案都是满足静态的负载均衡,而对于运行时(动态)的负载均衡
都没有涉及。